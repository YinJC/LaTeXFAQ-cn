\section{绘图篇}
%
%
%\begin{faq}{如何利用Tikz画超过360º的角,并做好标注}
%
%下面解答来自\url{https://tex.stackexchange.com/questions/60295/drawing-angles-greater-than-360\%C2\%BA-intikz}
%
%\begin{verbatim}
%\documentclass[11pt]{scrartcl}
%\usepackage{tikz}
%\usetikzlibrary{arrows}
%\begin{document}
%\newcommand\bigangle[2][]{%
%\draw[->,domain=0:#2,variable=\t,samples=200,>=latex,#1]
%plot ({(\t+#2)*cos(\t)/(#2)},
%{(\t+#2)*sin(\t)/(#2)}) node[right=.5cm] {$#2^\circ$}
%;}
%\begin{tikzpicture}
%\draw [thick] ( 0,0) -- (3,0);
%\draw [thick] ( 0,0) -- (0,3);
%\draw [red,thick] ( 0,0) -- (400:3);
%\bigangle[blue,dashed]{400}
%\end{tikzpicture}
%\end{document}
%\end{verbatim}
%
%% \includegraphics{https://i.stack.imgur.com/4RO8Y.png}{图片图片}
%\end{faq}
%
%
%\begin{faq}{如何画交换图?}
%\end{faq}
%
%
%\begin{faq}{如何在插入的图(\cs{includegraphics}\Arg{...})的特定位置插入符号或图}
%
%可以使用 overpic 宏包,在 overpic 环境内进行图形绘制,环境内可以使用
%latex 自生 picture 环境的绘图语句进行绘制。以下面给出一个例子:
%
%\begin{verbatim}
%\documentclass{article}
%\usepackage{graphicx}
%\usepackage{overpic}
%\begin{document}
%\begin{overpic}[width=0.8\textwidth
%%,grid,tics=10 % 取消注释可以产生网格线帮助定位,完成后再将其注释。
%]{1.png}
%\put(35,10){\LaTeX}
%\put(80,0){\includegraphics[width=0.16\textwidth]{1.png}}
%\end{overpic}
%\end{document}
%\end{verbatim}
%
%下面左边是原图,右边是代码处理后的图片:
%% \includegraphics{https://images-cdn.shimo.im/VFe2jSxjmoMZpr2B/1.png!thumbnail}
%% \includegraphics{https://images-cdn.shimo.im/4qswwF2P4u44JNYB/2.png!thumbnail}
%
%当然还有一种方法可以使用 tikz 宏包,在 tikzpicture
%环境引入图片,并在此基础上进行绘图,这样的优点是绘图语句更为丰富,功能更强大。下面举个例子:
%
%\begin{verbatim}
%![图片](https://images-cdn.shimo.im/LrJDsfrVZIAerfPW/image.png!thumbnail)  
%![图片](https://images-cdn.shimo.im/sVlNCmljor0SctYs/image.png!thumbnail)
%\end{verbatim}
%
%第一个为原图,第二个是在原图基础上添加一个标号和边.
%其中的格线是用来辅助做图的。
%
%\begin{verbatim}
%\documentclass{article}
%\usepackage{tikz}
%\begin{document}
%\pgfdeclarelayer{foreground}
%\pgfdeclarelayer{background}
%\pgfsetlayers{background,main,foreground}
%\begin{tikzpicture}
%\begin{pgfonlayer}{background}
%\path[xshift=-1.24pt,yshift=4pt]      (0,0) node (o) {
%\includegraphics[width=2.8cm]{graph.pdf}};
%\end{pgfonlayer}
%\begin{pgfonlayer}{foreground}
%\node at (0,0) [label={[label distance=-2mm]40:$u$}]{};
%\draw[step=.5cm,help lines] (-1.4,-1.4) grid (1.4,1.4);
%\coordinate (A) at (0,1.36);
%\coordinate (B) at (0.91,-1.14);
%\fill[color=red] (B) circle (1pt);
%\draw[line width=0.6pt](A)--(B);
%\end{pgfonlayer}
%\end{tikzpicture}
%\end{document}
%\end{verbatim}
%
%
%
