% Copyright (C) 2018 by latexstudio <http://www.latexstudio.net>
%
% This program is free software: you can redistribute it and/or modify
% it under the terms of the GNU General Public License as published by
% the Free Software Foundation, either version 3 of the License, or
% (at your option) any later version.
%
% This program is distributed in the hope that it will be useful,
% but WITHOUT ANY WARRANTY; without even the implied warranty of
% MERCHANTABILITY or FITNESS FOR A PARTICULAR PURPOSE.  See the
% GNU General Public License for more details.
%
% You should have received a copy of the GNU General Public License
% along with this program.  If not, see <http://www.gnu.org/licenses/>.
%

% !TEX root = ../latex-faq-cn.tex
% !TEX program = xelatex
% !TEX options = --shell-escape

\section{开发篇-含LaTeX3}

介绍宏开发技巧,宏包和模板类开发的常见问题。


\faq{在阅读已有的宏包或者文类时,遇到未知的命令应如何处理}{}

可以参照胡伟的《LaTeX2e文类和宏包学习手册》中的第四章-命令集注。


\faq{文档类中键值对的实现}{}

使用 xkeyval 宏包。
该宏包主要定义了以下四个命令
\begin{itemize}
  \item Ordinary keys

    最基础的键值对定义方式。说明文档中,这部分主要介绍的是 xkeyval 定义出来的宏的参数和命令。

  \item Command keys

    用于定义存放数据的宏,如美赛模板中运用了该命令来存放参赛队伍编号。

  \item Choice keys

    用于定义一个具有选择性的宏,例如以下宏
\begin{texcode}
  \define@choicekey{HTNotes.cls}{cvcolor}%
  [\val\ar]{green,orange,violet,blue}[blue]{
    \ifcase\ar\relax
      \definecolor{themecolor}{HTML}{92D14F}
      \definecolor{cvtext}{HTML}{92909B}
    \or
      \definecolor{themecolor}{RGB}{230,140,20}
      \definecolor{cvtext}{RGB}{100,100,100}
    \or
      \definecolor{themecolor}{RGB}{178,10,142}
      \definecolor{cvtext}{RGB}{100,100,100}
    \else
      \definecolor{themecolor}{RGB}{0,164,215}
      \definecolor{cvtext}{RGB}{100,100,100}
    \fi
  }
\end{texcode}
  该宏实现的是cvcolor的选择性,可以选择 green, orange, violet, blue 等参数,且 blue 为默认参数。

  \item Boolean keys

    用于定义布尔宏,用于设置开关代码。美赛模板中大部分使用的是该宏。
\end{itemize}

  具体用法请参考其说明文档。

  使用了 xkeyval 后,如何进行参数传递之类的呢?
\begin{texcode}
  \DeclareOptionX*{\PassOptionsToClass{\CurrentOption}{book}}
  \ProcessOptionsX\relax
  \LoadClass[twoside,11pt]{book}
\end{texcode}

  那么,如何定义默认参数呢?
\begin{texcode}
  \ExecuteOptionsX{
    cvcolor = {blue},
    ...
  }
\end{texcode}

  如何设置接口呢?
\begin{texcode}
  \newcommand{\HTset}[1]{\setkeys{HTNotes.cls}{#1}}
\end{texcode}

  如何检测宏是否有内容呢?并依次为基础设置开关命令?
\begin{texcode}
  % 宏定义为
  % \define@cmdkey{HTNotes.cls}[Book@]{Writer}
  \ifdefempty{\Book@Writer}{\@author}{\Book@Writer}
\end{texcode}
  该命令实现的是检测 \verb|\Book@Writer| 宏是否有内容,如果有则使用其存储的内容,无则使用 \verb|\@author| 存储的参数。

  用 xkeyval 提供的键值对实现方式,基本不会出现宏包冲突的问题,因为其前缀是由其特定的包含方式的,前提是选取适当。
