\section{介绍公式的常见问题。}
%\end{faq}
%
%
%\begin{faq}{\cs{ldots} 与...有什么区别?}
%
%重定义的难度不同、造成的间距也不同。推荐使用 {[}\ldots{}{]}。 见
%\url{https://www.zhihu.com/question/27589739/answer/37255728}
%\end{faq}
%
%
%\begin{faq}{如何让长公式自动断行?}
%
%长公式自动断行要看情况,如果是在行内模式,合理使用空格,一般可以在二元运算符处断行,如果是行间模式,推荐使用align类环境,在需要断行处添加
%\textbackslash{} 手动断行。
%\end{faq}
%
%
%\begin{faq}{公式希腊字符如何加粗?}
%
%希腊字母没有粗体,可以选择合适的数学字体。 可以使用 bm
%宏包将希腊字母加粗。
%\end{faq}
%
%
%\begin{faq}{极限符号下面有两个趋近该怎么写}
%
%直接给出例子:
%
%\begin{verbatim}
%\documentclass{article}
%\begin{document}
%\[ \lim_{n\to\infty\atop m\to\infty} \]
%\end{document}
%\end{verbatim}
%
%或者使用 \cs{substack},代码如下:
%
%\begin{verbatim}
%\documentclass{article}
%\usepackage{amsmath}
%\begin{document}
%\[ \lim_{\substack{n\to\infty\\ m\to\infty}} \]
%\end{document}
%\end{verbatim}
%
%效果如下:
%% \includegraphics{https://images-cdn.shimo.im/FCY4A1SeBIcwBCGT/双重极限.PNG!thumbnail}
%\end{faq}
%
%
%\begin{faq}{怎样在 LaTeX
%  中输入引号}
%
%左引号用 `(键盘1旁边那个键),右引号用 `。双引号也一样,``''。
%中文条件下可以直接用中文引号(这个与编码方式和中文支持方式有关的),会有自动配对(这个和编辑器以及输入法有关的),但是如果需要用到不配对引号的情况,需要使用通用方法。
%\end{faq}
%
%
%\begin{faq}{align环境默认是居中对齐吗?我在使用时,发现公式开始是居中的,后来却一直靠右断对齐,这是什么原因?}
%
%\sout{align 默认靠右对齐,所以通常加 \&
%  符号,让代码左对齐。验证一下以下代码:}
%
%\begin{verbatim}
%\begin{align}
%& \nabla \times H = J,\\
%& \nabla \times E = - \partial _t B,\\
%& \nabla \cdot B = 0.
%\end{align}
%\end{verbatim}
%
%再试试把 \& 去掉什么样。
%align采用的是奇偶对齐的方式,第一列右对齐,第二列左对齐,就这样右左右左依此类推,两列之间用\&分隔。
%\end{faq}
%
%
%\begin{faq}{中英文标点使用规则不是很明白,尤其在公式环境里,字体和间距差别都比较大。怎样才能让正文和公式的标点统一(形状和间隔)?}
%
%详见:
%\url{https://link.zhihu.com/?target=http\%3A//www.moe.gov.cn/ewebeditor/uploadfile/2015/01/13/20150113092346124.pdf}
%在导言区加类似命令可实现全文替换:
%
%\begin{verbatim}
%\catcode`\。=\active\newcommand{。}{. }
%\end{verbatim}
%
%或者使用 xeCJK 宏包的字符映射功能,调用 fullwidth-stop
%这一映射文件,将中文空心句号映射为实心句点:
%
%\begin{verbatim}
%\documentclass{article}
%\usepackage{xeCJK}
%\setCJKmainfont[Mapping= fullwidth-stop]{STSong}
%\begin{document}
%句号。
%\end{document}
%\end{verbatim}
%\end{faq}
%
%
%\begin{faq}{公式如何居左对齐,居右对齐?}
%
%公式居左对齐在基础文档类中由 fleqn
%选项控制,选择该选项后,正文公式均居左对齐,至于居右对齐,嗯,我没见过这么奇怪的格式。
%\end{faq}
%
%
%\begin{faq}{公式之后解释公式符号的文字,通常是 ``符号 ------ 解释''
%  这样的格式,我希望这段文字的格式是按破折号对齐,并且解释文字折行后悬挂缩进,怎样实现这样的格式?}
%
%方法很多,可以列表,可以align等环境。 这里给出一个使用自定义列表的例子:
%
%\begin{verbatim}
%\usepackage{ifthen}
%\newcounter{qlst}
%\newenvironment{EqDesc}[2][式中]{%
%\begin{list}{}
%{%
%\usecounter{qlst}
%\settowidth{\labelwidth}{#1,#2\ --- \ }
%\setlength{\labelsep}{0pt}
%\setlength{\leftmargin}{\labelwidth}
%\setlength{\rightmargin}{0em}
%\setlength{\parsep}{0ex}
%\setlength{\itemsep}{0ex}
%\setlength{\itemindent}{0em}
%\setlength{\listparindent}{0em}
%\renewcommand{\makelabel}[1]{\stepcounter{qlst}\ifthenelse{\value{qlst}>1}{\hfill ##1\ --- \ 
%}{#1,\hfill ##1\ --- \ }}
%}}%
%{\end{list}}%
%\end{verbatim}
%
%EqDesc
%环境有两个参数,第一个为可选参数,是解释公式符号前的引导词,默认是``式中'',第二个参数是样本符号,可以选择一个列表中宽度最大的符号。条目
% \cs{item} 有一个可选参数(实际使用是必选参数),内容是要说明的符号。使用如下:
%
%\begin{verbatim}
%\[ a^2+b^2=c^2 \]
%\begin{EqDesc}[其中]{$a$}
%\item[$a$] 三角形的一条直角边;
%\item[$b$] 三角形的另一条直角边;
%\item[$c$] 三角形的斜边。
%\end{EqDesc}
%\end{verbatim}
%\end{faq}
%
%
%\begin{faq}{行内公式的情况下如何让sum
%  prod这些运算符的上下标在头上和脚下?}
%
%这样处理行内公式的上下标会导致段落行距不整齐,不符合 LaTeX
%的审美。如果彻底放弃审美,可以使用 \cs{limits} 命令,如:
%
%\begin{verbatim}
%$\sum\limits_{i=1}^n \quad
%\prod\limits_\epsilon$
%\end{verbatim}
%\end{faq}
%
%
%\begin{faq}{如何将积分的上限标放在积分号的上下两侧?}
%
%积分号的上下限放置在积分号的右侧是英美国家和 LaTeX
%的排版习惯,通常无需处理。如果你很确定需要按照 ISO 80000-2:2009 或者 GB
%3102.11-93 的规定排版积分号,可以:
%
%\begin{enumerate}
%  \def\labelenumi{\arabic{enumi}.}
%  
%  \item
%  在调用 amsmath 宏包时添加 intlimits 选项;
%  \item
%  \texttt{\textbackslash{}def\textbackslash{}int\{\textbackslash{}intop\}}
%  \item
%  如果使用 unicode-math 宏包,
%\end{enumerate}
%
%\begin{verbatim}
%\removenolimits{%
%\int\iint\iiint\iiiint\oint\oiint\oiiint
%\intclockwise\varointclockwise\ointctrclockwise\sumint
%\intbar\intBar\fint\cirfnint\awint\rppolint
%\scpolint\npolint\pointint\sqint\intlarhk\intx
%\intcap\intcup\upint\lowint
%}
%\end{verbatim}
%\end{faq}
%
%
%\begin{faq}{如何自定义数学运算符,然后让下标放在脚下?}
%
%借助 amsmath 包的
%\cs{DeclareMathOperator*} 命令即可(需要注意加不加*是有区别的)。例如
%
%\begin{verbatim}
%\DeclareMathOperator*{\esssup}{ess\,sup}
%\end{verbatim}
%\end{faq}
%
%
%\begin{faq}{在数学公式中,编辑等式时,每一行需要等号和等号对其,这时使用了\textbackslash{}begin\{displaymath\}
%  \textbackslash{}begin\{split\}环境,但是呢,这些整体都是居中的,我想让式子靠左,怎么实现呢?}
%\end{faq}
%
%
%\begin{faq}{行列式变换过程中,我们一般是在中间的箭头上表示出变换的方式,如何才能在长箭头上打出多行内容?}
%\end{faq}
%
%
%\begin{faq}{如何输出反斜杠?}
%
%\begin{verbatim}
%\textbackslash \verb|\|
%\end{verbatim}
%\end{faq}
%
%
%\begin{faq}{对equation环境下的公式、图片编号按章节、小节进行重新定义}
%\end{faq}
%
%
%
